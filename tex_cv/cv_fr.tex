\begin{header}
    \fontsize{25 pt}{25 pt}\selectfont \textbf{Sebastian Lopez Silva}

    \vspace{7 pt}

    %\normalsize
    %\textit{Étudiant en double diplôme à Mines Paris PSL et Universidad Nacional de Colombia}

    \vspace{1 pt}

    \normalsize
    \mbox{\faIcon{map-marker-alt} Paris, France}%
    %\kern 5.0 pt%
    %\AND%
    %\kern 5.0 pt%
    %\mbox{\faIcon{envelope} %\hrefWithoutArrow{mailto:sebastian.lopez\_silva@etu.minesparis.psl.eu}{sebastian.lopez\_silva@etu.minesparis.psl.eu}}%
    \kern 5.0 pt%
    \AND%
    \kern 5.0 pt%
    \mbox{\faIcon{envelope} \hrefWithoutArrow{mailto:slopezsi@unal.edu.co}{slopezsi@unal.edu.co}}%
    \kern 5.0 pt%
    \AND%
    \kern 5.0 pt%
    \mbox{\faIcon{phone} \hrefWithoutArrow{tel:+33 07 59 32 58 83}{+33 07 59 32 58 83}}%
    \kern 5.0 pt%
    %\AND%
    \kern 5.0 pt%
    \mbox{\faIcon{linkedin} \hrefWithoutArrow{https://www.linkedin.com/in/slopezsi}{slopezsi}}%
    \kern 5.0 pt%
    \AND%
    \kern 5.0 pt%
    \mbox{\faIcon{github} \hrefWithoutArrow{https://github.com/Sebls}{Sebls}}%
    \kern 5.0 pt%
    \AND%
    \kern 5.0 pt%
    \mbox{\faIcon{home} \hrefWithoutArrow{https://Sebls.github.io}{Sebls.github.io}}%

    \vspace{7 pt}
    
\end{header}

\vspace{5 pt - 0.3 cm}

\section{Profil}
    \begin{onecolentry}
    \textit{Étudiant en ingénierie informatique poursuivant un double diplôme international entre l’Universidad Nacional de Colombia et Mines Paris PSL. Avec une expérience en développement logiciel et une solide base en mathématiques, je suis motivé par la création de solutions utiles et l’innovation technologique. Passionné par l’IA et l’apprentissage automatique, je suis disponible à partir de mai 2026 pour des opportunités en France ou à l’international.}
    \end{onecolentry}

\section{Formation}
    \begin{twocolentry}{
        2025 - Prévu 2027
    }
        \textbf{École des Mines de Paris - PSL}, M.Sc. en ingénierie et mathématiques appliquées\end{twocolentry}

    \begin{twocolentry}{
        2021 - Prévu 2027
    }
        \textbf{Universidad Nacional de Colombia}, Licence en ingénierie informatique (GPA : 4,74/5,0)\end{twocolentry}

\section{Expérience}
\begin{onecolentry}
    \begin{highlightsforbulletentries}
    
    \item \textbf{Engin AI} - \textit{Ingénieur logiciel} (avr. 2025 – août 2025) – Miami, États-Unis (à distance)
    \begin{itemize}

        \item Amélioration des flux de travail de création et de préparation de jeux de données pour des modèles d’IA audio et vidéo, avec 70–80\% de gains de performance.

        \item Mise en place de conversions, validations et débogages automatisés d’annotations entre YOLO et COCO, réduisant le contrôle qualité manuel de 15–18 étapes à 3–4.
        
        \item Déploiement de services d’inférence en production traitant des milliers de requêtes quotidiennes, incluant une mise à l’échelle automatique jusqu’à zéro pour réduire les coûts d’infrastructure.
        
        \item Contribution à l’entraînement et à l’évaluation de modèles de classification audio, augmentant la précision de 83\% à 91\%.
    \end{itemize}

    \item \textbf{Dataconstructors AI} - \textit{Ingénieur logiciel} (nov. 2024 – avr. 2025) – Bogotá, Colombie (à distance)
    \begin{itemize}

        \item Développement de pipelines back-end pour automatiser le traitement de fichiers géospatiaux pour des projets de génie civil, réduisant le temps de traitement de \(\sim\)1 semaine à \(\sim\)3 heures.

        \item Conception et implémentation d’automatisations pour la génération de rapports et des statistiques métier en temps réel, améliorant la vitesse de livraison d’environ \(\sim\)80\% dans les processus clients.
        
        \item Développement et maintenance de systèmes full-stack et back-end intégrant des API externes et des bases MySQL/PostgreSQL, en appliquant les principes de Clean Architecture.
        
    \end{itemize}

    \item \textbf{Intelligent Systems Lab (LISI)} – \textit{Assistant de recherche (IA)} (août 2023 – août 2025) – Bogotá, Colombie
    \begin{itemize}
        \item Collaboration sur des travaux de recherche appliquée en IA et des projets internes avec des chercheurs de licence et master.
        \item Animation de 10+ séminaires en apprentissage profond (théorie + TP) pour 30–40 étudiants de licence/master, couvrant les réseaux de neurones et les techniques fondamentales d’IA.
    \end{itemize}
    \end{highlightsforbulletentries}
\end{onecolentry}

\section{Projets}
\begin{onecolentry}
    \begin{highlightsforbulletentries}

    \item \textbf{Pipeline automatisé pour la génération de modèles en Gaussian Splatting} – \textit{Mines Paris PSL - MOVIE: Ingénierie des Mondes Virtuels} (Sep 2025 – Nov 2025)
    \begin{itemize}
        \item Développement d’une exposition virtuelle immersive d’artefacts géologiques avec le Gaussian Splatting pour améliorer le réalisme et la qualité visuelle.
        \item Conception d’un pipeline automatisé à \(\sim\)90\% pour générer des jeux de données et des modèles de Gaussian Splatting, réduisant le temps de préparation de \(\sim\)5 heures à \(\sim\)30 minutes (hors temps d’entraînement).
    \end{itemize}
    
    % \item \textbf{TransMilenio Congestion Prediction System using Temporal Graphs Neural Networks} - \textit{Universidad Nacional de Colombia - Neural Networks} (Nov 2024 – Feb 2025)
    % \begin{itemize}
    % \item Researched the application of neural networks on spatio-temporal graphs to analyze articulated bus occupancy data in Bogotá’s Bus Rapid Transit system, aiming to make short-term occupancy predictions.
%\end{itemize}

    \end{highlightsforbulletentries}
\end{onecolentry}

\section{Compétences}

    \begin{onecolentry}
        \textbf{Compétences transversales :} Travail d'équipe • Résolution de problèmes • Adaptabilité • Communication • Créativité • Pensée critique
    \end{onecolentry}

    \begin{onecolentry}
        \textbf{Programmation :} Python (FastAPI, Pandas, NumPy, PyTorch, TensorFlow, etc.) • C/C++ • Julia • Java • TypeScript (React) • SQL (PostgreSQL, MySQL, Sqlite) • NoSQL (Redis, MongoDB)
    \end{onecolentry}

    \begin{onecolentry}
        \textbf{Outils techniques :} Git • Docker • Amazon Web Services • Azure • Google Cloud Platform
    \end{onecolentry}

    \begin{onecolentry}
        \textbf{Langues :} Espagnol (natif) • Anglais (C1) • Français (B2) • Portugais (conversationnel)
    \end{onecolentry}

\section{Divers}

    \begin{twocolentry}{
        Présent
    }
    \textbf{Exonération des frais de scolarité pour l’année académique 2025–2026}, Mines Paris PSL
    \end{twocolentry}
    
    \begin{twocolentry}{
        Présent
    }
    \textbf{Exonération des frais sur toutes les périodes académiques ; récipiendaire du tableau d’honneur (2022)}, Universidad Nacional de Colombia
    \end{twocolentry}
    
    \begin{twocolentry}{
        2024
    }
        \textbf{Top 1\% du GPA parmi les étudiants universitaires à Bogotá (\(\approx\)40 universités)}, \hrefWithoutArrow{https://www.agenciaatenea.gov.co/premios-jovenes-la-e-2024}{Youth for Excellence}
    \end{twocolentry}

    \begin{twocolentry}{
        2024
    }
        \textbf{Concours ICPC UNAL}, participant au concours de programmation compétitive
    \end{twocolentry}