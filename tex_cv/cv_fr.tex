\begin{header}
    \fontsize{25 pt}{25 pt}\selectfont \textbf{Sebastian Lopez Silva}

    \vspace{1 pt}

    \normalsize
    \textit{Étudiant en double diplôme à Mines Paris PSL et Universidad Nacional de Colombia}

    \vspace{1 pt}

    \normalsize
    \mbox{\faIcon{map-marker-alt} Paris, France}%
    \kern 5.0 pt%
    \AND%
    \kern 5.0 pt%
    \mbox{\faIcon{envelope} \hrefWithoutArrow{mailto:sebastian.lopez\_silva@etu.minesparis.psl.eu}{sebastian.lopez\_silva@etu.minesparis.psl.eu}}%
    \kern 5.0 pt%
    \AND%
    \kern 5.0 pt%
    \mbox{\faIcon{envelope} \hrefWithoutArrow{mailto:slopezsi@unal.edu.co}{slopezsi@unal.edu.co}}%
    \kern 5.0 pt%
    \AND%
    \kern 5.0 pt%
    \mbox{\faIcon{phone} \hrefWithoutArrow{tel:+33 7 59 32 58 83}{+33 7 59 32 58 83}}%
    \kern 5.0 pt%
    \AND%
    \kern 5.0 pt%
    \mbox{\faIcon{linkedin} \hrefWithoutArrow{https://www.linkedin.com/in/slopezsi}{linkedin.com/in/slopezsi}}%
    \kern 5.0 pt%
    \AND%
    \kern 5.0 pt%
    \mbox{\faIcon{github} \hrefWithoutArrow{https://github.com/Sebls}{github.com/Sebls}}%
    \kern 5.0 pt%
    \AND%
    \kern 5.0 pt%
    \mbox{\faIcon{home} \hrefWithoutArrow{https://Sebls.github.io}{Sebls.github.io}}%
    
\end{header}

\vspace{5 pt - 0.3 cm}

\section{Profil}
\begin{onecolentry}
\textit{Étudiant en ingénierie informatique en double diplôme international entre l’Universidad Nacional de Colombia et l’École des Mines de Paris. Fort d’une première expérience en développement logiciel et d’un solide socle en mathématiques, je suis motivé par la création de solutions utiles et l’innovation technologique. Passionné par l’IA et le Machine Learning, je suis disponible à partir de mai 2026 pour des opportunités en France ou à l’international.}
\end{onecolentry}


\section{Formation}
    \begin{twocolentry}{
        Aoû 2025 – Jul 2027
    }
        \textbf{École des Mines de Paris – PSL}, Cycle d'Ingénieur Civil\end{twocolentry}

    \begin{twocolentry}{Oct 2021 – Sep 2027}
        \textbf{Universidad Nacional de Colombia}, Licence en Ingénierie Informatique\end{twocolentry}

    \begin{twocolentry}{
        Nov 2025
    }
        \textbf{Université Technique Tchèque de Prague}, Programme ATHENS - Algorithmes Randomisés\end{twocolentry}

\section{Expérience}
\begin{onecolentry}
    \begin{highlightsforbulletentries}
    
    \item \textbf{Engin AI} - \textit{Ingénieur Logiciel} (Avr 2025 – Aoû 2025) – Miami, États-Unis (Remote)
    \begin{itemize}
        \item Développement et maintenance d'outils de création, de prétraitement et de contrôle qualité de jeux de données pour des modèles de vision par ordinateur sur vidéo.

        \item Mise en place de flux automatisés pour convertir, valider et déboguer des annotations entre formats (YOLO, COCO) afin d'alimenter les pipelines d'entraînement.
        
        \item Support au diagnostic et à la résolution des problèmes d'intégrité et d'annotation issus de la plateforme interne.
        
        \item Contribution à des expériences d'entraînement de modèles (ex. RT-DETR) pour vérifier la cohérence des données et fiabiliser l'entraînement.
    \end{itemize}

    \item \textbf{Dataconstructors AI} - \textit{Ingénieur Logiciel} (Nov 2024 – Avr 2025) – Bogotá, Colombie (Remote)
\begin{itemize}
    \item Participation au développement de trois applications web : deux pour automatiser le traitement de fichiers géospatiaux en ingénierie civile (rôle backend) et une pour la gestion de projets et l'estimation des coûts (rôle full-stack).
\item Conception d'interfaces, de fonctionnalités d'automatisation des coûts et d'intégrations de données avec des APIs externes.
% \item Développement d'algorithmes de traitement de données externes, d'analyseurs de fichiers et de systèmes d'authentification.
\item Intégration et gestion de bases de données MySQL/PostgreSQL, en appliquant les principes d'architecture propre pour assurer scalabilité et maintenabilité.
\end{itemize}


    \item \textbf{Laboratoire de Systèmes Intelligents (LISI)} – \textit{Chercheur Étudiant} (Aoû 2023 – Présent) – Bogotá, Colombie
\begin{itemize}
    \item Membre actif du groupe de recherche, participation à des séminaires en IA et à des projets collaboratifs avec étudiants de licence et master.
    \item Préparation et présentation de supports sur des techniques et architectures de deep learning (CNN, VAE, Transformers, etc.).
    % \item Contribution à la rédaction de propositions soumises aux Appels Minciencias 950 et 966, intégrant IA et réalité virtuelle pour l'éducation environnementale.
    % \item Served as \textit{Student Research Assistant} during two research cycles (Jul–Dec 2024 and Jan–Aug 2025).
\end{itemize}
    \end{highlightsforbulletentries}
\end{onecolentry}

\section{Projets Universitaires}
\begin{onecolentry}
    \begin{highlightsforbulletentries}

    \item \textbf{Pipeline Automatisé pour la Génération de Modèles en Gaussian Splatting} – \textit{Mines Paris PSL - MOVIE: Ingénierie des Mondes Virtuels} (Sep 2025 – Nov 2025)
    \begin{itemize}
        \item Création d'une exposition virtuelle immersive d'artefacts géologiques à l'aide de Gaussian splats pour renforcer réflectance, translucidité et réalisme.
        \item Construction d'un pipeline automatisé intégrant AWS (S3 et Lambda), RealityScan et LightField Studio pour la génération de jeux de données et de modèles Gaussian Splatting.
    \end{itemize}

    
    

    % \item \textbf{TransMilenio Congestion Prediction System using Temporal Graphs Neural Networks} - \textit{Universidad Nacional de Colombia - Neural Networks} (Nov 2024 – Feb 2025)
    % \begin{itemize}
    % \item Researched the application of neural networks on spatio-temporal graphs to analyze articulated bus occupancy data in Bogotá’s Bus Rapid Transit system, aiming to make short-term occupancy predictions.
%\end{itemize}

    \end{highlightsforbulletentries}
\end{onecolentry}

\section{Compétences}

    
    \begin{onecolentry}
            \textbf{Langues :} Espagnol (natif) • Anglais (C1) • Français (B2) • Portugais (conversationnel)
    \end{onecolentry}

    \begin{onecolentry}
        \textbf{Compétences transversales :} Travail d'équipe • Résolution de problèmes • Adaptabilité • Communication • Créativité • Pensée critique
    \end{onecolentry}

    \begin{onecolentry}
        \textbf{Programmation :} Python (FastAPI, Pandas, NumPy, PyTorch, TensorFlow, etc.) • Julia • Java • C/C++ • Typescript (React) • SQL (PostgreSQL, MySQL, Sqlite) • NoSQL (Redis, MongoDB)
    \end{onecolentry}

    \begin{onecolentry}
        \textbf{Outils techniques :} Git • AWS • Azure • GCP • Jira • Wolfram Mathematica • LaTeX (Overleaf) • Microsoft Office
    \end{onecolentry}

\section{Distinctions}


    \begin{twocolentry}{
        Présent
    }
    \textbf{Exonération des frais de scolarité 2025–2026}, Mines Paris PSL\end{twocolentry}
    

    \begin{twocolentry}{
        Présent
    }
        \textbf{Exonération des frais pour toutes les périodes académiques}, Universidad Nacional de Colombia\end{twocolentry}
    
    \begin{twocolentry}{
        2024
    }
        \textbf{Reconnaissance : Meilleur GPA universitaire à Bogotá}, Premios Jóvenes a la Excelencia
    \end{twocolentry}
    
    \begin{twocolentry}{
        2024
    }
        \textbf{Reconnaissance : Meilleur GPA STEM à Bogotá}, Premios Jóvenes a la Excelencia
    \end{twocolentry}

    \begin{twocolentry}{
        2022
    }
        \textbf{Tableau d'honneur 2022-2}, Universidad Nacional de Colombia\end{twocolentry}

    \begin{twocolentry}{
        2020
    }
        \textbf{Reconnaissance : Résilience et dépassement personnel}, Colegio Rodrigo Arenas Betancourt\end{twocolentry}

\section{Bénévolat}
\begin{onecolentry}
    \begin{highlightsforbulletentries}
    \item \textbf{AspirantesVirtual} - \textit{Enseignant} (Jan 2025 – Jun 2025) – Colombie
    \begin{itemize}
        \item Conception de cours et tutorat en physique et mathématiques pour les candidats à l'Universidad Nacional de Colombia.
    \end{itemize}

    \item \textbf{Fondation Colombienne de Réalité Virtuelle} - \textit{Développeur} (Mar 2020 – Jun 2021) – Colombie
    \begin{itemize}
        \item Développement de fonctionnalités de jeu pour lever des fonds et soutien d'initiatives sociales en faveur des communautés vulnérables pendant la pandémie de COVID-19.
    \end{itemize}
    \end{highlightsforbulletentries}
\end{onecolentry}

\section{Loisirs}
    
    \begin{onecolentry}
        \textbf{Programmation compétitive :} Passionné par les problèmes algorithmiques et la participation à des compétitions. Membre d'une équipe ayant participé à l'événement ICPC Unal en juillet 2024.
    \end{onecolentry}

    \begin{onecolentry}
        \textbf{Jeux vidéo :} Joueur curieux explorant divers genres et plateformes, appréciant l'art graphique et la narration des jeux.
    \end{onecolentry}

